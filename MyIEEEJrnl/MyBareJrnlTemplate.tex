\documentclass[journal]{IEEEtran}


% ---------------------------------------------------------------------------------------------------------------------------------
% PACKAGES
% ---------------------------------------------------------------------------------------------------------------------------------
% *** CITATION PACKAGES ***
\usepackage{cite}

% *** GRAPHICS RELATED PACKAGES ***
\usepackage[dvips]{graphicx}
% \graphicspath{{../pdf/}{../jpeg/}}
% \DeclareGraphicsExtensions{.pdf,.jpeg,.png}
% \DeclareGraphicsRule{.tif}{png}{.png}{`convert #1 `basename #1 .tif`.png}

% *** MATH PACKAGES ***
\usepackage[cmex10]{amsmath}
\usepackage{amssymb}

% *** SPECIALIZED LIST PACKAGES ***
\usepackage{color}
%\usepackage{enumerate}
%\usepackage{algorithmic}

% *** ALIGNMENT PACKAGES ***
%\usepackage{array}
%\usepackage{mdwmath}
%\usepackage{mdwtab}
%\usepackage{eqparbox}

% *** FIGURE PACKAGES ***
\usepackage{psfrag}
\usepackage[tight,footnotesize]{subfigure}
%\usepackage{epsfig}
%\usepackage[caption=false]{caption}
%\usepackage[font=footnotesize]{subfig}

% *** FLOAT PACKAGES ***
%\usepackage{fixltx2e}
%\usepackage{stfloats}

% *** PDF, URL AND HYPERLINK PACKAGES ***
\usepackage{url}

% *** MISCELLANEOUS ***
%\hyphenation{op-tical net-works semi-conduc-tor}

% ---------------------------------------------------------------------------------------------------------------------------------
% MACROS
% ---------------------------------------------------------------------------------------------------------------------------------
\providecommand{\T}[0]{\textsf{T}} % Transpose
\providecommand{\abs}[1]{\lvert#1\rvert} % Absolute value
\providecommand{\norm}[1]{\lVert#1\rVert} % Norm
\providecommand{\N}{\mathbb{N}} % Natural Numbers
\providecommand{\Z}{\mathbb{Z}} % Inegers Numbers
\providecommand{\R}{\mathbb{R}} % Real Numbers
\newcommand{\note}{\textcolor{red}} % Notes

% ---------------------------------------------------------------------------------------------------------------------------------
% BEGIN DOCUMENT
% ---------------------------------------------------------------------------------------------------------------------------------
\begin{document}
\bibliographystyle{ieeetr}

% ---------------------------------------------------------------------------------------------------------------------------------
% TITLE AND AUTHORS
% ---------------------------------------------------------------------------------------------------------------------------------
\title{\note{My Bare Bones of IEEEtran Latex Template for Journals}}

\author{Rowland~O'Flaherty, \note{John~Doe,~\IEEEmembership{Fellow,~OSA,} and~Jane~Doe,~\IEEEmembership{Member,~IEEE}}% <-this % stops a space
\thanks{R. O'Flaherty is with the \note{???} Group, Georgia Institute of Technology, Atlanta, Georgia 30332, e-mail: rowland.oflaherty@gatech.edu.}% <-this % stops a space
\thanks{\note{J. Doe and J. Doe are with Anonymous University.}}% <-this % stops a space
\thanks{Manuscript received \note{MONTH DD, YYYY}; revised \note{MONTH DD, YYYY.}}}

% The paper headers
\markboth{Journal of \LaTeX\ Class Files,~Vol.~Number, No.~Number, Month~YYYY}%
{Shell \MakeLowercase{\textit{et al.}}: \note{My Bare Bones of Latex Template for Journals}}

% Use for special paper notices
%\IEEEspecialpapernotice{(Invited Paper)}

% Make the title area
\maketitle

% ---------------------------------------------------------------------------------------------------------------------------------
% ABSTRACT
% ---------------------------------------------------------------------------------------------------------------------------------
\begin{abstract}
\note{The abstract goes here}
\end{abstract}

% Use when submitting for peer review
%\IEEEpeerreviewmaketitle

% ---------------------------------------------------------------------------------------------------------------------------------
% KEYWORDS
% ---------------------------------------------------------------------------------------------------------------------------------
\begin{IEEEkeywords}
\note{Keyword1}, \note{Keyword2}, \note{Keyword3}.
\end{IEEEkeywords}

% Use when submitting for peer review
%\IEEEpeerreviewmaketitle

% ---------------------------------------------------------------------------------------------------------------------------------
% INTRODUCTION
% ---------------------------------------------------------------------------------------------------------------------------------
\section{Introduction}
\IEEEPARstart{T}{his} \note{template file is intended to serve as a ``starter file''
for IEEE journal papers produced under \LaTeX\ using IEEEtran.cls version 1.7 and later.}


% ---------------------------------------------------------------------------------------------------------------------------------
% Subsection
% ---------------------------------------------------------------------------------------------------------------------------------
\subsection{\note{Subsection Heading Here}}
note{Subsection text here}


% ---------------------------------------------------------------------------------------------------------------------------------
% Subsubsection
% ---------------------------------------------------------------------------------------------------------------------------------
\subsubsection{\note{Subsubsection Heading Here}}
note{Subsubsection text here}


% ---------------------------------------------------------------------------------------------------------------------------------
% CONCLUSION
% ---------------------------------------------------------------------------------------------------------------------------------
\section{Conclusion}
note{The conclusion goes here}

% ---------------------------------------------------------------------------------------------------------------------------------
% APPENDICES
% ---------------------------------------------------------------------------------------------------------------------------------
\appendices
\section{\note{Section Heading}}
\note{Appendix one text goes here}

\section{}
\note{Appendix two text goes here}

% Use this section for acknowledgement
\section*{Acknowledgment}
\note{The authors would like to thank...}


% ---------------------------------------------------------------------------------------------------------------------------------
% REFERENCES
% ---------------------------------------------------------------------------------------------------------------------------------
 \bibliography{BibCollectionName}

% ---------------------------------------------------------------------------------------------------------------------------------
% END DOCUMENT
% ---------------------------------------------------------------------------------------------------------------------------------
\end{document}


